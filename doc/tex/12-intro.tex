\Introduction

Среди большого числа понятий, которые возникли и исследуются в информатике и кибернетике, одним из наиболее важных является понятие алгоритма. Алгоритм представляет процесс решения задачи как последовательное выполнение простых (или ранее определенных) шагов. Каждое действие, предусмотренное алгоритмом, исполняется только после того, как закончилось исполнение предыдущего. Понятие алгоритма тесно связано с понятием конечного автомата, для них свойственен одинаковый способ функционирования: система переходит из состояния в состояние в соответствии с заданной функцией переходов и осуществляет очередной (последовательный) шаг алгоритма.

По мере усложнения решаемых задач все большее внимание привлекают "неалгоритмические" параллельные системы с недетерминированным поведением, в которых отдельные компоненты функционируют, в основном, независимо, иногда взаимодействуя друг с другом. Примером могут служить такие системы параллельной обработки информации, как многопроцессорные вычислительные машины, параллельные программы, моделирующие параллельные дискретные системы и их функционирование, мультипрограммные операционные системы и т.п.

Системы с параллельно функционирующими и асинхронно взаимодействующими компонентами не описываются адекватно в терминах классической теории автоматов. Такие фундаментальные понятия, как состояние автомата и глобальная функция перехода не удобны для наглядной и экономичной характеризации недетменированной динамики поведения систем с локальными связями между независимыми параллельными процессами.

Среди многих существующих методов описания и анализа дискретных параллельных систем выделился подход, который основан на использовании сетевых моделей, восходящий к сетям особого вида, предложенным Карлом Петри для моделирования асинхронных информационных потоков в системах преобразования данных.

Часто алгоритм функционирования сложных систем представляют в виде диаграммы деятельности UML, отличающейся от классической блок-схемы наличием элементов для представления многопоточной обработки. При этом в многопоточных системах встают вопросы о достижимости состояний и наличии блокировок. Для решения этих вопросов часто диаграмму деятельности представляют в виде простой сети Петри, при этом моделируется лишь процесс выполнения без привязки к данным. В отличие от простых сетей Петри, в раскрашенных сетях немаловажную роль играет типизация данных, основанная на понятии множества цветов, которое аналогично типу в декларативных языках программирования. Таким образом, представив диаграмму в виде раскрашенной сети Петри можно провести моделирование работы с учетом типов входных данных, что максимально приблизит процесс моделирования к реальному функционированию процесса.

Целью работы является разработка метода представления диаграммы деятельности раскрашенной сетью Петри, позволяющего выявить блокировки и недостижимые состояния.

Для достижения поставленной цели необходимо решить следующие задачи:
\begin{itemize}
\item[1.] классифицировать существующие методы анализа диаграмм деятельности;
\item[2.] разработать метод представления диаграммы деятельности в виде раскрашенной сети Петри;
\item[3.] программно реализовать метод представления диаграммы деятельности в виде раскрашенной сети Петри;
\item[4.] исследовать факторы, влияющие на появление блокировок;
\item[4.] исследовать корректность построения раскраски сети.
\end{itemize}
